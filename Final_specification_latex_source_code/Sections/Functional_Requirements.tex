\customchapter{Requirements}
\begin{longtable}{|p{3.5cm}|p{1cm}|p{2cm}|p{2cm}|p{6.5cm}|}
 \hline
 \textbf{Requirement} & \textbf{ID} & \textbf{Importance} & \textbf{Verifiable} & \textbf{Description} \\
 \hline
 \hline
Register file (peripheral) & G01 & High & VHDL-TB & An IP with 32x 32-bit registers is designed to be interfaced as a peripheral with the ARM Cortex-A9 core. This register file acts as an slave.
\\
 \hline
  Write access & G02 & High & VHDL-TB & Register file (peripheral) provides write access to the other surrounding peripherals.
\\
\hline
Read access & G03 & High & VHDL-TB & Register file (peripheral) provides read access to the other surrounding peripherals.
\\
\hline
Communication between ARM Cortex-A9 core and Register file & G04 & High & VHDL-TB & AXI4-Lite bus is utilized to communicate between ARM Cortex-A9 Core and memory mapped Register file (peripheral).
 \\
 \hline
 AXI4-Lite channels & G05 & High & VHDL-TB & All the five channels of AXI4-Lite bus are essential for communication.
\\
\hline
 Memory mapping of register file (peripheral) & G06 & High & VHDL-TB/ Application program & The register peripheral must be mapped to certain addresses among the memory map of the ARM Cortex-A9 core.
\\
\hline
 RISC V compliance & G07 & High & Application program & The register file (peripheral) must be RISC V compliant, i.e., all RISC V instructions should be able to access it (Ex: R-Type instruction, I-Type instruction etc). 
  \\
 \hline
 Max registers in Register file (peripheral) & G08 & High &  & There can be only 32 maximum registers within the register file, as RISC V instruction layout only supports 5 bits for addressing both source and destination registers .\\
 \hline
 Word size & G09 & High & VHDL-TB & Word size of the ARM Cortex-A9 core is 32 bit, hence the each register within the register file (peripheral) is of 32 bits wide.
  \\
 \hline
 Register file access test & G10 & High & Application Program & Two registers of register file (peripheral) are interfaced with Z-Board LEDs and switch from a cross-compiled assembly/C program.
 \\
 
  \hline
 Scan Test & G11 & Medium & JTAG & TAP controller is designed to perform scan tests on the register file (peripheral).
\\
 \hline
 Clock access to the TAP controller & G12 & Medium & JTAG & TAP controller needs a clock signal to operate the TAP controller state machine. TCLK pin of the TAP controller provides this clock input to the state machine from the JTAG network.
  \\

 \hline
 Control signal for the TAP controller & G13 & Medium & JTAG & TAP controller needs a control signal to control the TAP controller state machine. TMS pin of the TAP controller provides this control signal to the state machine from the JTAG network.
  \\
 \hline
 Instruction data to TAP controller & G14 & Medium & JTAG & TDI pin of TAP controller forwards the instruction data from the JTAG network to the state machine.
  \\
 \hline
 Input test data to TAP controller & G15 & Medium & JTAG & TDI pin of TAP controller forwards the input test data from the JTAG network to the register file.
  \\
 \hline
 Collect output test data from the TAP controller & G16 & Medium & JTAG & TDO pin of TAP controller collects the output test data from the register peripheral and then sends it to the JTAG network for observation.
  \\
 \hline
 Transfer test data from TAP controller to register file (peripheral) & G17 & Medium & JTAG & Test data from the TDI of TAP controller can be sequentially cascaded into each flipflop of the register file (peripheral).
\\
 
 \hline
Collect test data from register file (peripheral) into TAP Controller & G18 & Medium & JTAG & Cascaded test data from the register file (peripheral) will be cascaded back into the JTAG network via TDO of TAP controller.
  \\
  \hline
  
  Processor core & G19 & High & Z-Board & ARM Cortex-A9 core present on the Z-Board is used as the main CPU to interface the designed peripheral. This processor core acts as an master. \\
 \hline
\end{longtable}
